\section{Постановка задачи}

    В данной работе мы рассматриваем общую задачу распределённой оптимизации:
    \begin{equation}
        \min \limits_{x \in \R^d}  \left\{f(x) := \frac{1}{n} \sum \limits_{i=1}^n f_i(x) \right\}, \label{eq:problem_statement}
    \end{equation}
    где $x \in \R^d$ представляет параметры модели, по которым происходит оптимизация, $n$ обозначает количество устройств, $f_i(x)$~--- локальная функция потерь для устройства $i$. На функции $f_i$ накладываются условия $\mu$-сильной выпуклости и $L$-гладкости.

    \begin{definition}
        Функция $f$ называется $\mu$-сильно выпуклой, если для любых $x, y \in \R^d$ выполняется следующее неравенство:
        \begin{equation}
            f(y) \geq f(x) + \langle \nabla f(x), y - x \rangle + \frac{\mu}{2} \|y - x\|^2, \label{eq:strong_convexity}
        \end{equation}
        где $\mu > 0$~--- константа сильной выпуклости.
    \end{definition}

    \begin{definition}
        Функция $f$ называется $L$-гладкой, если для любых $x, y \in \R^d$ выполняется следующее неравенство:
        \begin{equation}
            \|\nabla f(y) - \nabla f(x)\| \leq L \|y - x\|, \label{eq:smoothness}
        \end{equation}
        где $L > 0$~--- константа гладкости.
    \end{definition}


    Для решения задачи федеративного обучения используется следующая модель коммуникации:

    \begin{enumerate}
        \item Все устройства параллельно вычисляют градиент или стохастический градиент своей локальной функции потерь $f_i$.
        \item Каждое устройство отправляет вычисленный градиент на центральное устройство~--- сервер.
        \item Сервер обрабатывает полученные градиенты, выполняет шаг оптимизации и передаёт обновлённые параметры модели обратно всем устройствам.
    \end{enumerate}

    Этот процесс повторяется до тех пор, пока не будет достигнут критерий остановки, например, достижение заданного порога функции потерь $f$ или выполнение заданного числа итераций.

    Для уменьшения затрат на коммуникацию при пересылке градиентов используются операторы сжатия. Приведем определения двух классов операторов, которые будут использоваться в данной работе.

    \begin{definition}
        Пусть $\zeta \geq 1$. Мы говорим, что $\mathcal{C} \in \mathcal{U}(\zeta)$, если $\mathcal{C}$ является несмещённым (то есть $\mathbb{E}[\mathcal{C}(x)] = x$ для всех $x$) и если его второй момент ограничен следующим образом:
        \begin{equation}
            \mathbb{E}\left[\|\mathcal{C}(x)\|_2^2\right] \leq \zeta \|x\|_2^2, \quad \forall x \in \R^d.
        \end{equation}
    \end{definition}

    \begin{definition}
        Пусть $\delta > 0$. Мы говорим, что $\mathcal{C} \in \mathcal{B}(\delta)$, если выполняется следующее условие:
        \begin{equation}
            \mathbb{E}\left[\|\mathcal{C}(x) - x\|_2^2\right] \leq \left(1 - \frac{1}{\delta}\right) \|x\|_2^2, \quad \forall x \in \R^d.
        \end{equation}
    \end{definition}

    